% Article Format: American Economic Review
% Xin Tang @ Stony Brook University
% Last Updated: August 2014
\documentclass[twoside,11pt,leqno]{article}

%%%%%%%%%%%%%%%%%%%%%%%%%%%%%%%%%%%%%%%%%%%%%%%%%%%%%%%%
%                    Text Layout                       %
%%%%%%%%%%%%%%%%%%%%%%%%%%%%%%%%%%%%%%%%%%%%%%%%%%%%%%%%
% Page Layout
\usepackage[hmargin={1.2in,1.2in},vmargin={1.5in,1.5in}]{geometry}
\topmargin -1cm        % read Lamport p.163
\oddsidemargin 0.04cm   % read Lamport p.163
\evensidemargin 0.04cm  % same as oddsidemargin but for left-hand pages
\textwidth 16.59cm
\textheight 21.94cm
\renewcommand\baselinestretch{1.15}
\parskip 0.25em
\parindent 1em
\linespread{1}

% Set header and footer
\usepackage{fancyhdr}
\pagestyle{fancy}
\fancyhead{}
\fancyhead[LE,RO]{\thepage}
%\fancyhead[CE]{\textit{JI QI, XIN TANG AND XICAN XI}}
%\fancyhead[CO]{\textit{THE SIZE DISTRIBUTION OF FIRMS AND INDUSTRIAL POLLUTION}}
\cfoot{}
\renewcommand{\headrulewidth}{0pt}
%\renewcommand*\footnoterule{}
%\setcounter{page}{1}

% Font
\renewcommand{\rmdefault}{ptm}
\renewcommand{\sfdefault}{phv}
%\usepackage[lite]{mtpro2}
% use Palatinho-Roman as default font family
%\renewcommand{\rmdefault}{ppl}
\usepackage[scaled=0.88]{helvet}
\makeatletter   % Roman Numbers
\newcommand*{\rom}[1]{\expandafter\@slowromancap\romannumeral #1@}
\makeatother
\usepackage{CJK}

% Section Titles
\renewcommand\thesection{\textnormal{\textbf{\Roman{section}.}}}
\renewcommand\thesubsection{\textnormal{\Alph{subsection}.}}
\usepackage{titlesec}
\titleformat*{\section}{\bf \center}
\titleformat*{\subsection}{\it \center}
\renewcommand{\refname}{\textnormal{REFERENCES}}

% Appendix
\usepackage[title]{appendix}
\renewcommand{\appendixname}{APPENDIX}

% Citations
\usepackage[authoryear,comma]{natbib}
\renewcommand{\bibfont}{\small}
\setlength{\bibsep}{0em}
\usepackage[%dvipdfmx,%
            bookmarks=true,%
            pdfstartview=FitH,%
            breaklinks=true,%
            colorlinks=true,%
            %allcolors=black,%
            citecolor=blue,
            linkcolor=red,
            pagebackref=true]{hyperref}

% Functional Package
\usepackage{enumerate}
\usepackage{url}      % This package helps to typeset urls

%%%%%%%%%%%%%%%%%%%%%%%%%%%%%%%%%%%%%%%%%%%%%%%%%%%%%%%%
%                    Mathematics                       %
%%%%%%%%%%%%%%%%%%%%%%%%%%%%%%%%%%%%%%%%%%%%%%%%%%%%%%%%
\usepackage{amsmath,amssymb,amsfonts,amsthm,mathrsfs,upgreek}
% Operators
\newcommand{\E}{\mathbb{E}}
\newcommand{\e}{\mathrm{e}}
\DeclareMathOperator*{\argmax}{argmax}
\DeclareMathOperator*{\argmin}{argmin}
\DeclareMathOperator*{\plim}{plim}
\renewcommand{\vec}[1]{\ensuremath{\mathbf{#1}}}
\newcommand{\gvec}[1]{{\boldsymbol{#1}}}

\newcommand{\code}{\texttt}
\newcommand{\bcode}[1]{\texttt{\blue{#1}}}
\newcommand{\rcode}[1]{\texttt{\red{#1}}}
\newcommand{\rtext}[1]{{\red{#1}}}
\newcommand{\btext}[1]{{\blue{#1}}}

% New Environments
\newtheorem{result}{Result}
\newtheorem{assumption}{Assumption}
\newtheorem{proposition}{Proposition}
\newtheorem{lemma}{Lemma}
\newtheorem{corollary}{Corollary}
\newtheorem{definition}{Definition}
\setlength{\unitlength}{1mm}

%%%%%%%%%%%%%%%%%%%%%%%%%%%%%%%%%%%%%%%%%%%%%%%%%%%%%%%%
%                 Tables and Figures                   %
%%%%%%%%%%%%%%%%%%%%%%%%%%%%%%%%%%%%%%%%%%%%%%%%%%%%%%%%
\usepackage{threeparttable,booktabs,multirow,array} % This allows notes in tables
\usepackage{floatrow} % For Figure Notes
\floatsetup[table]{capposition=top}
\usepackage[font={sc,footnotesize}]{caption}
\DeclareCaptionLabelSeparator{aer}{---}
\captionsetup[table]{labelsep=aer}
\captionsetup[figure]{labelsep=period}
\usepackage{graphicx,pstricks,epstopdf}

%%%%%%%%%%%%%%%%%%%%%%%%%%%%%%%%%%%%%%%%%%%%%%%%%%%%%%%%
%                 Insert Code Snippet                  %
%%%%%%%%%%%%%%%%%%%%%%%%%%%%%%%%%%%%%%%%%%%%%%%%%%%%%%%%
\usepackage{listings,textcomp,upquote}
\lstset{
     language=fortran,
     frame = single,
%     backgroundcolor=\color[RGB]{255,228,202}, % pink
     backgroundcolor=\color[RGB]{231,240,233}, % green
%     backgroundcolor=\color[RGB]{239,240,248},
     framerule=0pt,
     showstringspaces=false,
     basicstyle=\ttfamily\footnotesize,
     numbers=left,
     stepnumber=1,
     numberstyle=\tiny,
     keywordstyle=\color{blue}\ttfamily,
     stringstyle=\color{red}\ttfamily,
     commentstyle=\color[rgb]{.133,.545,.133}\ttfamily,
     morecomment=[l][\color{magenta}]{\#},
     fontadjust,
     captionpos=t,
     framextopmargin=2pt,framexbottommargin=2pt,
     abovecaptionskip=4ex,belowcaptionskip=3pt,
     belowskip=3pt,
     framexleftmargin=4pt,
     xleftmargin=4em,xrightmargin=4em,
     texcl=false,
     extendedchars=false,columns=flexible,mathescape=true,
     captionpos=b,
}
\renewcommand{\lstlistingname}{Source Code}

\title{\vspace{-1cm}\Large{{\textsf{Using Modern Programming Tools in Macroeconomics}}}}
\author{\normalsize\textsc{Xin Tang} \\ \normalsize\textsc{International Monetary Fund}}
\date{\normalsize\today}

\begin{document}
\maketitle

Modern quantitative macroeconomics relies heavily on solving models numerically. Moving beyond the realm of homework in graduate school, it is rarely the case that the program solving a model contains only several small files with no reliance on any third-party libraries at all. In addition, as the average scale of academic projects increases, nowadays a research project usually requires close collaboration among several researchers. The time when a simple command
\begin{verbatim}
$ gfortran main.f90 utility.f90 -o main.out -Ofast
\end{verbatim}
generates the executable \code{main.out} which consequently computes all the results fades distantly day by day.

A typical program nowadays usually contains a large number of files each focuses on one task, draws heavily on well-tested numerical libraries, and involves several contributing researchers simultaneously. Computer scientists and software engineers have developed many tools that help make the development process of projects alike as seamless as possible. It is a bit unfortunate that these tools and the corresponding programming and software development paradigm were never introduced properly for an average graduate student in economics.

To fill this gap, this note provides an introduction to a collection of commonly used open-source programming tools and numerical libraries available on most Linux-based systems. Specifically, they are
\begin{itemize}
    \item
    \code{gfortran}: an open-source Fortran compiler from \textsc{Gcc}.
    \item
    \code{make}: a build automation tool facilitates building executable file from a large number of source files and libraries.
    \item
    \code{git}: a version-control system for tracking changes in source code during software development.
    \item
    \code{BLAS, ATLAS, LAPACK, MINPACK}: standard libraries for optimized numerical linear algebra and nonlinear optimization.
\end{itemize}

The note is organized as follows. We begin with an example of solving a system of nonlinear equations. This will familiarize the reader with programming under a Linux system. Using this example, we then provide a very brief overview of how source code is turned from textual files to an executable file, which helps us introduce \code{make} and explain how to use libraries distributed in binary format. Here, we will solve a system of linear equations calling routines in \code{LAPACK}, which is distributed in binary format. Finally, we show how to use \code{git} to track the history of changes of the source code, to revert the stage of a messy project to a previous status, and to collaborate effectively with other researchers without worrying about the pains and hassles caused by things like overwriting each other's work. For the last part, because I am currently refactoring the code of \citet{Azzimontietal:2014} with modern coding conventions, so I will use its development process as an example. Maybe I will consider crafting a simpler example in the future.

\section{Compiling a Sophisticated Project}

\subsection{Using MINPACK}

Consider the following example of solving a system of nonlinear equations in $\vec{x} = [x_1, x_2, x_3]$:
\begin{align*}
    f_1(\vec{x}) &= x_1 + \e^{x_1-1} + (x_2+x_3)^2 - 27 \\
    f_2(\vec{x}) &= \e^{x_2-2}/x_1 + x_3^2 - 10 \\
    f_3(\vec{x}) &= x_3 + \sin(x_2-2) + x_2^2 - 7.
\end{align*}
It is easy to verify that the true solution is $\vec{x} = [1,2,3]$. Code \ref{lst:hybrd}, which is saved as the text file \newline \code{./src/minpack\_ex.f90}, solves the problem.
\lstinputlisting[caption={Example: \code{hybrd}},label={lst:hybrd}
]{./src/minpack_ex.f90}
Similarly, the source code of \code{MINPACK} is saved as the text file \code{./lib\_src/minpack.f90}. Therefore, to compile the program, we invoke the following command in Shell
\begin{verbatim}
$ gfortran ./src/minpack_ex.f90 ./lib_src/minpack.f90 -o minpack_ex.out -Ofast
\end{verbatim}
This generates an executable file \code{minpack\_ex.out} under the root directory. Execute the program, the terminal prints the following information
\begin{verbatim}
fnorm =  2.3536E-007
 info =            1
    x =  1.0000    2.0000    3.0000
\end{verbatim}

Now you may notice that by the above command, every time we compile the source code to generate the executable, \code{minpack.f90} is compiled again. This can be very time consuming if the library is large and optimization at compiling stage is applied. It would make much more sense to just compile \code{minpack.f90} once given that we do not change anything at all later when calling any routines inside, and later just recompile the code that calls the routines in \code{MINPACK}. This is done by compiling files independently.

In particular, with option \code{-c}, \code{gfotran} can turn the source code into a binary object file. When object files are supplied as input, \code{gfortran} automatically invokes the \textit{linker} to patch the object files together and generate an executable file. Hence in practice, only the object files that have been modified need to be recompiled. Specifically, we use the following shell commands to generate two corresponding \code{.o} object files, save them under the subdirectory \code{./obj/}, and then patch them together to generate the executable:
\begin{verbatim}
$ gfortran -c ./src/minpack_ex.f90 -o ./obj/minpack.ex.o -Ofast
$ gfortran -c ./lib_src/minpack.f90 -o ./obj/minpack.o -Ofast
$ gfortran ./obj/minpack_ex.o ./obj/minpack.o -o minpack_ex.out -Ofast
\end{verbatim}
The next time when we change anything \textit{only} in \code{minpack\_ex.f90}, we only need to recompile itself but not \code{minpack.f90}.

\subsection{Understand How Linker Works}

Better though, we can use the archiver to turn a number of object files into one \code{.a} library file. To build a library from \code{minpack.o}, use the following command
\begin{verbatim}
$ ar cr libminpack.a ./obj/minpack.o
\end{verbatim}
where the option \code{cr} stands for \textit{create} and \textit{replace}. Now if you use any archiver or data compression program to open \code{libminpack.a}, you will find the text file \code{1.txt}, which contains all the functions defined by \code{minpack.f90}. The content of \code{1.txt} is shown is Code \ref{lst:minpack}.
\lstinputlisting[caption={The List of Routines in MINPACK},label={lst:minpack}
]{./obj/1.txt}
In practice, linker uses the information in \code{1.txt} to resolve symbol reference.

Since a library is nothing but a collection of \code{.o} object files, it can be used in the linking stage in pretty much the same way as ordinary object files:
\begin{verbatim}
$ gfortran ./src/minpack_ex.f90 ./obj/libminpack.a -o minpack_ex.out -Ofast
\end{verbatim}
where again
\begin{verbatim}
$ ./minpack_ex.out
\end{verbatim}
delivers the same results.

\subsection{Using \code{LAPACK}, \code{BLAS}, and \code{ATLAS}}

Now we can explain how \code{LAPACK}, \code{BLAS}, and \code{ATLAS} can be used in practice since these libraries are distributed in both static (\code{.a}) and dynamic library (\code{.so}) formats.

First, you need to install these libraries in your system. I am using Ubuntu under WSL, so the commands are
\begin{verbatim}
$ sudo apt-get install libblas-dev
$ apt-get install libatlas-base-dev
$ sudo apt-get install liblapack-dev
\end{verbatim}
for \code{BLAS}, \code{ATLAS}, and \code{LAPACK} respectively.

Now let us solve the example of OLS in IMSL. The objective is to do a linear regression with three independent variables
\begin{equation*}
    y_i = \beta_0 + \beta_1 x_{1i} + \beta_2 x_{2i} + \beta_3 x_{3i} + \varepsilon_i,
\end{equation*}
on data
\begin{equation*}
    \vec{X} = \begin{bmatrix}
        7.0  & 5.0  & 6.0 \\
        2.0  & -1.0 & 6.0 \\
        7.0  & 3.0  & 5.0 \\
        -3.0 & 1.0  & 4.0 \\
        2.0  & -1.0 & 0.0 \\
        2.0  & 1.0  & 7.0 \\
        -3.0 & -1.0 & 3.0 \\
        2.0  & 1.0  & 1.0 \\
        2.0  & 1.0  & 4.0 \end{bmatrix} \qquad
    \vec{Y} = \begin{bmatrix}
        7.0 \\ -5.0 \\ 6.0 \\ 5.0 \\ 5.0 \\ -2.0 \\ 0.0 \\ 8.0 \\ 3.0 \end{bmatrix}
\end{equation*}
Using any standard statistical software, it is very easy to verify that the OLS estimate is
\begin{equation*}
    \gvec{\upbeta} = \begin{bmatrix} 7.73 & -0.20 & 2.33 & -1.67 \end{bmatrix}.
\end{equation*}
Now let us use \code{LAPACK} to solve the problem. To do this, we need to call the routine \code{DGELS}. Code \ref{lst:lapack} does the job.
\lstinputlisting[caption={Example: \code{LAPACK}},label={lst:lapack}
]{./src/lapack_ex.f90}

To compile the file, we use the following command
\begin{verbatim}
$ gfortran ./src/lapack_ex.f90 -llapack -latlas -lblas \
    -Ofast -o lapack_ex.out
\end{verbatim}
where \code{\textbackslash} indicates a break of the line.

\section{Version Control and Collaboration}

To initialize a \code{Git} repository, first create a repository on GitHub. Here I use my personal account and create a repository called \code{intro\_tool}. The url of the repository is
{\color{blue}\begin{verbatim}
https://github.com/zjutangxin/intro_tool
\end{verbatim}}
\noindent Next we add it as a new remote repository and name it as \code{origin} locally:
\begin{verbatim}
$ git clone https://github.com/zjutangxin/intro_tool
\end{verbatim}
which fetches everything on the remote repository and labels the remote repository as \code{origin}. We then copy a bunch of files to the local working directory and add files to the tracking list
\begin{verbatim}
$ git add ./**/*.f90
  git add ./*.tex
\end{verbatim}
We can then see the status of the current branch by
\begin{verbatim}
$ git status -uno
\end{verbatim}
where the option \code{-uno} suppresses the display of untracked files. We then commit all the files by
\begin{verbatim}
$ git commit -m 'initialize local files'
\end{verbatim}
Notice that, each commit must always come with a committing message, otherwise \code{Git} aborts the current commit.

With all the preparation, we push all the files to the remote repository by
\begin{verbatim}
$ git push origin master
\end{verbatim}
which will then ask for the user name and credentials for the GitHub account where you have write permission. When executed, the terminal prints the following message:
\begin{verbatim}
> Username for 'https://github.com': zjutangxin
  Password for 'https://zjutangxin@github.com':
  Counting objects: 10, done.
  Delta compression using up to 8 threads.
  Compressing objects: 100% (10/10), done.
  Writing objects: 100% (10/10), 579.00 KiB | 0 bytes/s, done.
  Total 10 (delta 1), reused 0 (delta 0)
  remote: Resolving deltas: 100% (1/1), done.
  To https://github.com/zjutangxin/intro_tool
     319c5e7..c91de67  master -> master
\end{verbatim}

Now if you go to the online portal of GitHub, you should see the local files are all synchronized. From now on, anytime you have modified some files and would like to save the current status of the whole project, simply type
\begin{verbatim}
$ git add <fliename.extension>
\end{verbatim}
to stage the files you want to commit (case sensitive), and
\begin{verbatim}
$ git commit -m 'commit message'
\end{verbatim}
to commit all the changes. Notice that sometimes after you make a commit, immediately you may realize that you forgot to include a file or two, or you have made a typo in the commit message, you do not want to flood your commit history with small commit like this. In such situation, use the \code{--amend} option to replace the most recent commit using the current one:
\begin{verbatim}
$ git commit --amend -m 'commit message'
\end{verbatim}
However, this way, your local branch will diverge from the remote branch, normal push request would be rejected as a result. As a result, the \code{-f} option should be used to force push the request
\begin{verbatim}
$ git push origin master -f
\end{verbatim}

\bibliography{D:/Dissertation/Literature/Dissertation1}
\bibliographystyle{aea}

\end{document}
